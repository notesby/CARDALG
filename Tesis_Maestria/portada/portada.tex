% - - - - - - - - - - - - - - - - - - - - - - - - - - - - - - - - - - - - -
% portada.tex-- Portada para tesis de posgrado de la Facultad de Ingeniería. UNAM.
% Realizada por Gengis Kanhg Toledo Ramírez (gengiskanhg.geo@yahoo.com)
% Basada en la portada escrita por: Tim Rohrer, LT, USN--31 July 1996 del NPS, Monterey, California. USA.
% y en el manual The Not So Short Introduction to LaTeX de Tobiaz Oetiker
% Eres libe de modificar este archivo de acuerdo a tus necesidades.
% 13 de Junio de 2007
% 
% Compile with:
% pdflatex portada.tex
% - - - - - - - - - - - - - - - - - - - - - - - - - - - - - - - - - - - - -
% Modificada por Julio Fernando Jimenez Vielma  CIC-IPN
% 2011
% - - - - - - - - - - - - - - - - - - - - - - - - - - - - - - - - - - - - -
% Adaptación para la maestría MCIC-CIC-IPN
% Alejandro Gómez <alegzc@yahoo.com.mx>
% 2012
% - - - - - - - - - - - - - - - - - - - - - - - - - - - - - - - - - - - - -
\documentclass{book}
%\usepackage[latin1]{inputenc} ¿used in Windows?
\usepackage[utf8]{inputenc} % Linux's text encoding scheme
\usepackage[spanish]{babel}
\usepackage{graphicx}

\setlength{\voffset}{-0.5cm}
\setlength{\hoffset}{0.7cm}
\setlength{\headsep}{0pt}
\setlength{\headheight}{0pt}
\setlength{\oddsidemargin}{-0.8in}
\setlength{\marginparwidth}{-0.5cm}
\setlength{\textwidth}{19.5cm}
\setlength{\footskip}{2pt}
\setlength{\topmargin}{0in}
\setlength{\textheight}{25cm}
\setlength{\fboxrule}{3pt}

\begin{document}
\thispagestyle{empty}

\begin{tabular}{p{3cm}p{15.0cm}}
\includegraphics[width=2.5cm]{fig/ipn.png}
\begin{center}
\rule[1cm]{1.5mm}{14.5cm}%vertical
\hspace{2pt}
\rule[0cm]{0.7mm}{15.5cm}%vertical
\hspace{2pt}
\rule[1cm]{1.5mm}{14.5cm}%vertical
\end{center}
\includegraphics[width=3cm]{fig/cic.png}
&
\vspace{-3.4cm}
\begin{center}
\Large{ \bf{INSTITUTO POLITÉCNICO NACIONAL}}
\\
\rule[0mm]{15.0cm}{0.35mm}%horizontal
\\
\rule[3mm]{15.0cm}{1.2mm}%horizontal
\\
\textbf{CENTRO DE INVESTIGACIÓN EN COMPUTACIÓN}

\vspace{2.8\baselineskip}

Laboratorio de Inteligencia Artificial \\
Laboratorio de Simulación y Modelado

\vspace{2.3\baselineskip}

{\Large \bf{Aprendizaje de fenómenos con representación de estados en 2D}}

%\vfill
\vspace*{1.2cm}

\huge{\bf TESIS}

\vspace*{1.0cm}
{\large QUE PARA OBTENER EL GRADO DE:}

\vspace*{0.2cm}

\Large{\bf MAESTRO EN CIENCIAS DE LA COMPUTACIÓN}

%\vspace*{0.2cm}
%\large{MEC�NICA - DISE�O MEC�NICO}

\vspace*{0.2cm}
P R E S E N T A:

\vspace*{1.0cm} {\Large \bf{Ing. Héctor Daniel Moreno Leyva}}

\vspace*{1.0cm}
Directores de tesis:

\large{\bf Dr. Salvador Godoy Calderón\\M. C. Germán Téllez Castillo}

\vspace*{1.8cm}
\Large{Mexico, CDMX}\hspace*{5cm}\Large{Octubre 2020}

\end{center}

\end{tabular}

%\end{center}
\newpage % La razón de esta página en blanco es la de mantener el orden (izq, derecha)
%          en el resto del documento al que se adjuntará...
\thispagestyle{empty}
$ $ % junk
\end{document}