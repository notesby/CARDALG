% + + + + + + + + + + + + + + + + + + + + + + + + + + + + + + + + + +
% + + + + + + + + + + + + + + + + + + + + + + + + + + + + + + + + + +
% +          _             _                           _            +
% +   __ _  | |__    ___  | |_   _ __    __ _    ___  | |_          +
% +  / _` | | '_ \  / __| | __| | '__|  / _` |  / __| | __|         +
% + | (_| | | |_) | \__ \ | |_  | |    | (_| | | (__  | |_          +
% +  \__,_| |_.__/  |___/  \__| |_|     \__,_|  \___|  \__|         +
% +                                                                 +
% + + + + + + + + + + + + + + + + + + + + + + + + + + + + + + + + + +
% + + + + + + + + + + + + + + + + + + + + + + + + + + + + + + + + + + 
\pagestyle{plain}
\pagenumbering{roman}
\cleardoublepage
\chapter*{Resumen}
\addcontentsline{toc}{chapter}{Resumen}
\bigskip
\noindent Para la modelación de sistemas físicos, químicos y biológicos, los autómatas celulares han sido una herramienta importante por su capacidad de modelar sistemas dinámicos de manera discreta en el espacio y tiempo; y a su vez también ha sido una herramienta cuyo uso es limitado por la complejidad que existe en la generación de las reglas de evolución que modela el sistema completamente. Se ha demostrado que los algoritmos de aprendizaje simbólicos y sub-simbólicos han resultado de gran utilidad para generar modelos interpretables a partir de un conjunto de datos, es por esto que en el presente trabajo se estudio, cómo usar estas herramientas de modelado, diseñando e implementando un algoritmo de aprendizaje automático que permita generar reglas a partir de un conjunto de datos representativos obtenidos de cierto fenómeno, sin perder la semántica que se encuentra embebida en los datos de los que se está aprendiendo.

% + + + + + + + + + + + + + + + + + + + + + + + + + + + + + + + + + +
% + + + + + + + + + + + + + + + + + + + + + + + + + + + + + + + + + +
%\vspace*{1cm}
\bigskip
\bigskip
%\cleardoublepage
\chapter*{Abstract}
\addcontentsline{toc}{chapter}{Abstract}
\bigskip
The modelation of physical, biological and chemistry systems, the cellular automata models have been an important tool for their capacity to model dynamical systems in a discrete way in space and time; also it is a tool whose use is limited by the complexity that exists to select the right evolve transition rules to model the system completely. It has been demonstrated that symbolic and sub-symbolic algorithms of learning have been of great utility to generate interpretative models from a dataset, is because of this that in order to facilitate the use of this tools, we are proposing a model that let us generate a set of rules from a representative dataset of a certain phenomena, without loosing the semantics of what does the data represent.
