% + + + + + + + + + + + + + + + + + + + + + + + + + + + + + + + + + +
% + + + + + + + + + + + + + + + + + + + + + + + + + + + + + + + + + +
% +          _             _                           _            +
% +   __ _  | |__    ___  | |_   _ __    __ _    ___  | |_          +
% +  / _` | | '_ \  / __| | __| | '__|  / _` |  / __| | __|         +
% + | (_| | | |_) | \__ \ | |_  | |    | (_| | | (__  | |_          +
% +  \__,_| |_.__/  |___/  \__| |_|     \__,_|  \___|  \__|         +
% +                                                                 +
% + + + + + + + + + + + + + + + + + + + + + + + + + + + + + + + + + +
% + + + + + + + + + + + + + + + + + + + + + + + + + + + + + + + + + + 
\pagestyle{plain}
\newcounter{savepage}
\pagenumbering{roman}
%\cleardoublepage
\chapter*{Resumen}
\addcontentsline{toc}{chapter}{Resumen}
\bigskip
\noindent Los autómatas celulares han sido una herramienta importante para el modelado de sistemas físicos, químicos y biológicos, debido a que tienen la capacidad de modelar sistemas dinámicos de manera discreta tanto en el tiempo como en el espacio. Sin embargo, esta herramienta tiene un uso limitado por la complejidad que existe en la generación de las reglas de evolución que modela el sistema completamente. Se ha demostrado que los algoritmos de aprendizaje simbólicos y sub-simbólicos han resultado de gran utilidad para generar modelos interpretables a partir de un conjunto de datos. Por esta razón, en el presente proyecto de investigación se estudió cómo usar estas herramientas de modelado, diseñando e implementando un algoritmo de aprendizaje automático que permita generar reglas a partir de un conjunto de datos representativos obtenidos de cierto fenómeno, sin perder la semántica que se encuentra embebida en los datos de los que se está aprendiendo.

% + + + + + + + + + + + + + + + + + + + + + + + + + + + + + + + + + +
% + + + + + + + + + + + + + + + + + + + + + + + + + + + + + + + + + +
%\vspace*{1cm}
%\bigskip
%\bigskip
%\cleardoublepage
\chapter*{Abstract}
\addcontentsline{toc}{chapter}{Abstract}
\bigskip
Cellular automaton have been an important tool for the modeling of physical, chemical and biological systems, because of their capacity to model dynamic systems discretely in space and time. However, the use of this tool is limited by the complexity that exists in the selection of the right evolve transition rules to model the system completely. It has been demonstrated that symbolic and sub-symbolic learning algorithms have been very useful to generate interpretative models from a data set. For this reason, in order to facilitate the use of these tools, we are proposing a model that generates a set of rules from a representative data set of a certain phenomena, without loosing the semantics of what does the data represent.
