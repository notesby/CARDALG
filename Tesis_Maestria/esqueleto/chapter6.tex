\chapter{Conclusiones y trabajo futuro}
En base a los resultados obtenidos en los experimentos realizados, podemos concluir que el algoritmo LRDEA es capaz de aprender a partir del conjunto de datos brindado y que lo hace con un porcentaje de exactitud que sobrepasa al algoritmo GA-Nuggets, sin embargo como vemos en las figuras \ref{fig:lrdeabrain}, \ref{fig:lrdeabyl}, \ref{fig:lrdeaevoloops} y \ref{fig:lrdeamite} que todavía es posible mejorar el algoritmo debido a que puede llegar a tener variaciones muy grandes entre la exactitud dentro del conjunto del entrenamiento y fuera de entrenamiento, en comparación con el algoritmo RA1 que es tiene un comportamiento mas estable. 

\section{Trabajo a futuro}

Como trabajo a futuro, se propone realizar la experimentación utilizando otro conjunto de datos, como por ejemplo: imágenes aéreas de áreas urbanas para el aprendizaje de reglas que simulen el crecimiento de población. Investigar una mejor función de aptitud, que nos ayude a reducir la variación entre la exactitud dentro y fuera de entrenamiento.