\chapter{Complete source code}
The listing \ref{hw-plat-mem-test} show the C code for the memory-base test on the hardware platform.\\[0.5cm]

%\begin{center}
%\begin{minipage}{10cm}
\lstset{%
	numbers=left, %
	numberstyle=\tiny, %
	stepnumber=1, %
	numbersep=10pt, %
	language=C,%
	breaklines=true,                % sets automatic line breaking
	breakatwhitespace=true,
	basicstyle=\footnotesize
}
\begin{lstlisting}[title={Memory-base test in C code},label=hw-plat-mem-test, caption={[Prueba de la plataforma hardware base]Prueba de la plataforma hardware base donde se ejercita el módulo DDR y el microprocesador principalmente.}]
#include "xparameters.h"
#include "xcache_l.h"
#include "stdio.h"
#include "xutil.h"
#include "xuartns550_l.h"

int main (void) {
   XCache_EnableICache(0xc0000000);
   XCache_EnableDCache(0xc0000000);

   /* Initialize RS232_Uart_1 - Set baudrate and number of stop bits */
   XUartNs550_SetBaud(XPAR_RS232_UART_1_BASEADDR, XPAR_XUARTNS550_CLOCK_HZ, 9600);
   XUartNs550_mSetLineControlReg(XPAR_RS232_UART_1_BASEADDR, XUN_LCR_8_DATA_BITS);
   print("-- Entering main() --\r\n");

   /* 
    * MemoryTest routine will not be run for the memory at 
    * 0xffff0000 (xps_bram_if_cntlr_1)
    * because it is being used to hold a part of this application program
    */


   /* 
    * MemoryTest routine will not be run for the memory at 
    * 0xfc000000 (FLASH)
    * because it is a read-only memory
    */


   /* Testing Memory (DDR2_SDRAM)*/
   {
      XStatus status;

      print("Starting MemoryTest for DDR2_SDRAM:\r\n");
      print("  Running 32-bit test...");
      status = XUtil_MemoryTest32((Xuint32*)XPAR_DDR2_SDRAM_MEM_BASEADDR, 1024, 0xAAAA5555, XUT_ALLMEMTESTS);
      if (status == XST_SUCCESS) {
         print("PASSED!\r\n");
      }
      else {
         print("FAILED!\r\n");
      }
      print("  Running 16-bit test...");
      status = XUtil_MemoryTest16((Xuint16*)XPAR_DDR2_SDRAM_MEM_BASEADDR, 2048, 0xAA55, XUT_ALLMEMTESTS);
      if (status == XST_SUCCESS) {
         print("PASSED!\r\n");
      }
      else {
         print("FAILED!\r\n");
      }
      print("  Running 8-bit test...");
      status = XUtil_MemoryTest8((Xuint8*)XPAR_DDR2_SDRAM_MEM_BASEADDR, 4096, 0xA5, XUT_ALLMEMTESTS);
      if (status == XST_SUCCESS) {
         print("PASSED!\r\n");
      }
      else {
         print("FAILED!\r\n");
      }
   }

   print("-- Exiting main() --\r\n");
   XCache_DisableDCache();
   XCache_DisableICache();
   return 0;
}
\end{lstlisting}
%\end{minipage}
%\end{center}

\clearpage


The listing \ref{hw-plat-per-test} show the C code for the peripheral-based test on the hardware platform.\\[0.5cm]

%\begin{center}
%\begin{minipage}{10cm}
\lstset{%
	numbers=left, %
	numberstyle=\tiny, %
	stepnumber=1, %
	numbersep=10pt, %
	language=C,%
	breaklines=true,                % sets automatic line breaking
	breakatwhitespace=true,
	basicstyle=\footnotesize
}
\begin{lstlisting}[title={Peripheral test C Code},label=hw-plat-per-test, caption={[Prueba de la plataforma hardware con periféricos]Prueba de la plataforma hardware base donde se ejercitan los periféricos en especial los Leds y los Dip-Switches.}]
a
\end{lstlisting}

