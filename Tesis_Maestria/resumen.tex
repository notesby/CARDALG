\chapter*{RESUMEN}\label{resumen}
\markboth{RESUMEN}{} % para que cambie el encabezado, si no, usar�a el del �ltimo chapter{}
\addcontentsline{toc}{chapter}{RESUMEN} % para que se a�ada en el indice

\baselineskip 1em
Principalmente para auxiliar a personas con problemas de habla, algunas \linebreak[4]investigaciones han implementado t�cnicas de aprendizaje m�quina para reconocer el habla directamente de la actividad cerebral. La electroencefalograf�a es una t�cnica no invasiva que se ha utilizado con este prop�sito. En este trabajo se describen algunas metodolog�as seguidas para probar si es factible reconocer el habla de se�ales electroencefalogr�ficas clasificando categor�as fonol�gicas binarias a trav�s de \linebreak[4]experimentos independientes de sujeto.\\

Estos experimentos se realizaron para dos actividades mentales diferentes: habla pronunciada e imaginada (pronunciaci�n externa e interna de sonidos o palabras, respectivamente). Por lo tanto, las muestras de cada actividad mental fueron utilizadas por clasificadores de dos enfoques diferentes: 1. basado en vectores y \linebreak[4]2. Espacio-temporales. Para el primer enfoque fueron utilizados clasificadores \linebreak[4]tradicionales, mientras que para el segundo enfoque se utilizaron clasificadores basados en neuronas pulsantes. Para estos experimentos, fueron necesarios pasos para el procesamiento de los datos y la extracci�n de caracter�sticas.\\

Para ambas actividades mentales (habla pronunciada e imaginada), los resultados obtenidos con el clasificador basado en una s�la neurona punsante super� a los dem�s en todos los experimentos. Adem�s, los mejores resultados obtenidos para las muestras del habla pronunciada fueron con el enfoque basado en vectores, mientras que para las muestras del habla imaginada fueron con el enfoque Espacio-temporal. Discuciones, an�lisis, y conclusiones sobre estos resultados se hacen al final de este trabajo.\\