%\centerline{\underline{\bf ABSTRACT}}
\chapter*{ABSTRACT}\label{abstract}
\markboth{ABSTRACT}{} % para que cambie el encabezado, si no, usar�a el del �ltimo chapter{}
\addcontentsline{toc}{chapter}{ABSTRACT} % para que se a�ada en el indice

\baselineskip 1em

Mainly to aid people with speaking problems, some researches have implemented machine learning techniques to recognize speech directly from brain activity. The electroencephalography is one non-invasive technique that has been used for this purpose. In this work are described some methodologies followed to test if it is feasible to recognize speech from electroencephalography signals by classifying binary phonological categories with subject-independent experiments.\\

These experiments were performed for two different mental activities: overt and imagined speech (external and internal pronunciation of sounds or words, respectively). Thus, the samples of each mental activity were used by classifiers from two different approaches: 1. Vector-based, and 2. Spatio-temporal. For the first approach, traditional classifiers were used, while for the second approach were used classifiers based on spiking neurons. For these experiments, specific data processing and feature extraction steps were necessary.\\

For both mental activities (overt and imagined speech), the scores obtained with the single spiking neuron classifier outperformed others in all the experiments. Besides, the best results obtained for overt speech samples were with the Vector-based approach, while for imagined speech samples were with the Spatio-temporal approach. \linebreak[4]Discussions, analysis, and conclusions from these results are made at the end of this work.

