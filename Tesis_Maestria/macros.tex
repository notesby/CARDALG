%---------------------------------------------------
%  Macros para LaTeX.
% Maximino Pena Guerrero.
%---------------------------------------------------
%---------------------------------------------------
% tit -Texto por pagina (acetatos).
%---------------------------------------------------
\newcommand{\tpp}[3]{
   \pagestyle{myheadings}
   \markright{
   \Huge\sc{#1}
   {\small Max.P.G., {#2}, {#3}}}
   \huge\sffamily}
%---------------------------------------------------
% \paginavacia -para capitulos que terminan non.
%---------------------------------------------------
\newcommand{\paginavacia}{
   \newpage
   \centerline{}
   \vspace{4in}
   \centerline{\small{$\Phi$}}
   \newpage
   }
%---------------------------------------------------
% ppa -Parrafo para acetatos.
%---------------------------------------------------
\newcommand{\ppa}[1]{
    \paragraph{
    \Huge\sffamily
    \bfseries{#1}}
    \Huge\sffamily\bfseries
  }
%---------------------------------------------------
% Contador de items
%---------------------------------------------------
\newcounter{nT}
\newcommand{\resetn}{\setcounter{nT}{0}}
\newcommand{\n}{\addtocounter{nT}{1}(\arabic{nT}) }
%---------------------------------------------------
% Contador de parrafos
%---------------------------------------------------
\newcounter{nP}                         % pP -Contador de Parrafo.
\setcounter{nP}{0}                      % pP -Inicio de Contador de Parrafo.
\newcommand{\ppcc}[1]{\addtocounter{nP}{1}
	   \paragraph{\arabic{nP}. #1}} % pcc -Parrafo con contador.
%---------------------------------------------------
% Medidas para fichas de trabajo, laser HP4050
%---------------------------------------------------
\newcommand{\whpnet}{ % ficha impresora laser 4050.
   \setlength{\textheight}{4.25in}
   \setlength{\textwidth}{7in}
   \setlength{\oddsidemargin}{0.25in}
   \setlength{\topmargin}{-2.5in}
   \setlength{\headsep}{0in}
   \setlength{\headheight}{3.25in}
   }
%---------------------------------------------------
\newcommand{\idea}[1]{\marginpar{\footnotesize #1}}   % pp -Parrafo Normal.
\newcommand{\npp}[1]{\paragraph{#1}}    % pp -Parrafo Normal.
\newcommand{\sen}[1]{\underline{#1}}    % sen -senal subrallada.
\newcommand{\sx}[1]{\shortstack{#1}}    % texto verical.
\newcommand{\di}{$\backslash$}          % di -Diagonal invertida.
\newcommand{\italic}[1]{{\it #1}}       % Texto en italica.
\newcommand{\txtit}[1]{{\it #1}}        % Texto en italica.
\newcommand{\tipo}[1]{{\tt #1}}         % Texto en teletipo.
\newcommand{\bface}[1]{{\bf #1}}        % Texto en boldface.
\newcommand{\fcons}{$2^{\frac{1}{12}}$} % fcons -Constante de tonos musicales.
\newcommand{\ub}[1]{\underbrace{#1}}    % corchete inferir.
\newcommand{\ob}[1]{\overbrace{#1}}     % corchete superior.
\newcommand{\NOT}[1]{\overline{#1}}     % NOT.
\newcommand{\OR}{+}                     % OR.
\newcommand{\AND}{\cdot}                % AND.
\newcommand{\laa}{\leftarrow}           % flecha izquierda.
\newcommand{\tlinea}{\hrule\vspace{0.5em}} % linea superior.
\newcommand{\dlinea}{\vspace{0.5em}\hrule} % linea inferior.
\newcommand{\ra}{\rightarrow}            % flecha derecha.
\newcommand{\lgra}{\Longrightarrow}     % flecha grande derecha.
\newcommand{\sub}[1]{\underline{#1}}    % corchete inferir.
\newcommand{\delay}{e^{-\tau s}}
\newcommand{\grad}{^{\circ}}
\newcommand{\ie}{\emph{i.e.}}
\newcommand{\real}{\mathbb{R}}
%---------------------------------------------------
% Encabezado de fichas
%---------------------------------------------------
\newcommand{\tleft}[2]{\leftline{\underline{\large\tt #1}\footnote{#2}}}
\newcommand{\tright}[2]{\rightline{\underline{\large\tt #1}\footnote{#2}}}
\newcommand{\comillas}[1]{\char92{#1}\char34}
\newcommand{\pregunta}[1]{\char62{#1}\char63}
\newcommand{\pg}[1]{(p: \pageref{#1})}
%---------------------------------------------------
%\def\I{{\'{\i}}}                        % I -i acentuada.
\def\ibid{{\it ib\'{\i}dem}}            % ibidem: ahi mismo, en el mismo lugar..
\def\idem{{\it \'{\i}dem}}              % idem: el mismo nombre.
\def\ejem{$e.g.$~}                      % por ejemplo ejempli grattia.
\def\nn{\nonumber}                      % no numerar ecuacion.

%---------------------------------------------- 